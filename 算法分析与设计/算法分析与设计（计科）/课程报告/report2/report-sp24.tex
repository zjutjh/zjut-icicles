
\usepackage{tikz}% 画图用的包
\usepackage{graphicx} %插入图片的宏包
\usepackage{float} %设置图片浮动位置的宏包
\usepackage{subfigure} %插入多图时用子图显示的宏包
\usepackage{fancyhdr} %设置页眉页脚的宏包
\usepackage[]{caption2} %设置图和表的格式的宏包
\usepackage{multirow} %合并多行单元格的宏包
\usepackage{longtable} %不宽但很长的表格可以用longtable宏包来进行分页显示
\usepackage{array} %一般用于数学公式中对数组或矩阵的排版
\usepackage{makecell}% makecell命令对表格单元格中的数据进行一些变换的控制。我们可以使用 \ 命令进行换行,也可以使用p{(宽度)}选项控制列表的宽度
\usepackage{threeparttable} %制作三线表格
\usepackage{booktabs}%s三线表格中的上中下直线线型设置宏包,在这个包中水平线被教程\toprule、midrule、buttomrule。
\usepackage{enumerate} %列举宏包

% 以下是伪代码
\usepackage{algorithm}
\usepackage[noend]{algpseudocode}
\floatname{algorithm}{算法}


%页眉页脚设置
\pagenumbering{arabic}
\pagestyle{fancy}
\setlength{\headheight}{15pt}
\fancyhead[L]{\reportType}
\fancyhead[R]{\className \reportSemester}
\fancyhead[C]{}
\fancyfoot[C]{\thepage}

%标题序号长度设置
\setcounter{secnumdepth}{3}

%图片排版设置
\renewcommand{\figurename}{图} %重定义编号前缀词
\renewcommand{\captionlabeldelim}{.~} %重定义分隔符
 %\roman是罗马数字编号,\alph是默认的字母编号,\arabic是阿拉伯数字编号,可按需替换下一行的相应位置
\renewcommand{\thesubfigure}{(\roman{subfigure})}%此外,还可设置图编号显示格式,加括号或者不加括号
\makeatletter \renewcommand{\@thesubfigure}{\thesubfigure \space}%子图编号与名称的间隔设置
\renewcommand{\p@subfigure}{} \makeatother

%表头文字格式的详细设置
\renewcommand\theadset{\renewcommand\arraystretch{0.85}%
\setlength\extrarowheight{0pt}}%行距
\renewcommand\theadfont{\small}%字体
\renewcommand\theadalign{rt}%行列对齐
\renewcommand\theadgape{\Gape[0.5cm][2mm]}%上下垂直距离

\title{
  \begin{figure}[H]
    \centering
    \includegraphics[width=1\textwidth]{./img/university.png}
  \end{figure} 
  \vspace{3em}
  \huge \textbf{\reportName} \\ 
  \vspace{1em}
  \large \textbf{\className-\reportType}\\
  \vspace{5em}
  \large 学生姓名\hspace{0.7cm}\underline{\makebox[5.5cm]{\studentName}} \\
  \large 学生学号\hspace{0.7cm}\underline{\makebox[5.5cm]{\studentID}} \\
  \large 专业班级\hspace{0.7cm}\underline{\makebox[5.5cm]{\studentGrade}} \\
  \large 指导教师\hspace{0.7cm}\underline{\makebox[5.5cm]{\prof}} \\
  \large 提交日期\hspace{0.7cm}\underline{\makebox[5.5cm]{\the\year 年 \the\month 月 \the\day 日}} \\
}

\author{}
\date{}

\ctexset { section = { format={\Large \bfseries } } }

